\documentclass[11pt, oneside]{article}   	% use "amsart" instead of "article" for AMSLaTeX format
\usepackage{geometry}                		% See geometry.pdf to learn the layout options. There are lots.
\geometry{letterpaper}                   		% ... or a4paper or a5paper or ... 
%\geometry{landscape}                		% Activate for rotated page geometry
%\usepackage[parfill]{parskip}    		% Activate to begin paragraphs with an empty line rather than an indent
\usepackage{graphicx}				% Use pdf, png, jpg, or eps§ with pdflatex; use eps in DVI mode
								% TeX will automatically convert eps --> pdf in pdflatex		
\usepackage{amssymb}
\usepackage{chicago}
\usepackage{hyperref}
\setlength\parindent{24pt}
\title{Homework 3}
\author{Lucas Cadalzo, Lorenz Gahn, Kenny Groszman (Live and Let Liver)}
\date{}							% Activate to display a given date or no date

\begin{document}
\maketitle

Hepatocellular carcinoma (HCC) causes about 660,000 deaths per year, with the majority occurring in developing nations. The prognosis for patients with the disease is poor, with the median life expectancy between 6-20 months. Curative therapies are not available to more than 80\% of patients. As a result, treatment, which is mostly nonsurgical, focuses on extending patients' lives while ensuring an acceptable quality of life. Currently, doctors do not have a good understanding of what tumor characteristics make a patient receptive or resistant to a certain treatment. As a result, treatment of HCC is largely a trial-and-error process, in which different treatment types are tried out until one appears to work. This process is inefficient and leads to shorter patient life expectancies. This project involves using machine learning to predict the effectiveness of different treatments for new HCC patients, so that doctors can determine the most suitable course of action. \par
One of the most common types of treatment for HCC is transarterial embolization (TASE), which involves inserting a catheter in a patient's hepatic artery, the blood supply of most liver tumors. The artery is then blocked by injecting small particles from the catheter. This approach spares healthy liver cells because they are usually fed by the portal vein. In the initial stages of this project, we are focusing on patients treated with TASE, although the approach will be extended to other treatments later on. \par
The aim of this project is to analyze CT scans and identify certain liver and tumor characteristics that correlate with responsiveness to TASE. If successful, this project will improve upon clinical practice for treatment of HCC. Instead of successively testing out therapies until a successful treatment has been found, doctors will be able to run an algorithm that informs them of the most effective treatment type. This personalized approach will save significant time and money that is currently being wasted on ineffective treatments. In destroying tumor cells, cancer treatment also often harm surrounding tissues. By eliminating ineffective treatment options, this algorithm will reduce damage to healthy organs, thereby increasing patients' quality of life. 


\end{document} 